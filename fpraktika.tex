\documentclass[12pt]{article}

\usepackage[T2A]{fontenc}
\usepackage[UTF8]{inputenc}
\usepackage{mathtools}
\usepackage[english,russian]{babel}
\usepackage[left=2cm,right=2cm,top=1cm,bottom=2cm]{geometry}

\begin{document}

\begin{center}
\bf \Large MathCad.
\end{center}
1. Ввести матрицу \it A \rm --- размерности $3 \times 4$, где $a_{ij} = \sin (0.3i+2j)$. Посчитать выражение
\[\it A \rm \ast 
\begin{pmatrix} 
& 5 & \\ 
& 4 & \\ 
& 6 & \\ 
& 4 & 
\end{pmatrix}
\ast 
\begin{pmatrix}
& 7 & -2 & 0.3 &
\end{pmatrix}
-
\begin{pmatrix}
& 1 & 0 & 2 &\\
& -3 & 2 & 6 &\\
& 0 & 0 & 3 &\\
\end{pmatrix}
.
\]
Высчитать определитель данного выражения.\\
2. На одном рисунке построить графики функций:
\begin {align*}
f_1(x)=e^{sin(\frac{x}{2})}, \quad f_2(x)=cos(x)\cdot sin(x)+2, \quad x \in [3,6]
\end {align*}
3. Построить график функции
\begin {align*}
e^{sin(x) \ast cos(y)}
\end {align*}
4. Написать программный модуль, для вычисления функции \it f(x, y) \rm --- разложение в ряд\linebreak Тейлора функции $ \sqrt{x} $, в окрестностях точки \it y\rm. Сделать проверку на некоректные входные\linebreak данные.\\
5. Вычислить
\begin {gather*}
\int \frac{(t-3)\sqrt{t+1}}{t^2}dt;\\
\frac{d}{dm}\biggl[e^{m^2} \ast \tg(\frac{3}{m})\biggr];\\
a=\sin(4);\\
a=\frac{\sqrt{5 \ast \sin(34)}}{|\sin(28 \ast 0.4 - 15\cos(3))|};\\
b=\tg(4) \ast \int\limits_{-1}^2  \sin(t^3)e^{4+t}dt;\\
a+b;
\end{gather*}
6.  Раскрыть скобки в выражении\\ 
\begin {equation*}
(x_3+3)^2 \ast (x_2-x_3)^3;
\end {equation*}
 Разложить в ряд $$e^x$$
\begin{center}
\bf \Large MatLab.
\end{center}
Все вычисления оформить в один M-файл.\\
7. Ввести матрицы\\
\begin {center}
$ A =
\begin{pmatrix} 
& 8 & 5 & 1 &\\
& -3 & -5 & 0 &\\
& 56 & 12.67 & 3.09 &\\
\end{pmatrix}
$
\quad и \quad
$ B =
\begin{pmatrix} 
& -4 & 0.4 & 21.8 &\\
& -0 & 67.02 & 12.96 &\\
& 4 & 45.28 & -3 &\\
\end{pmatrix}
$
\end {center}
Посчитать произведение матриц, разницу матриц, определитель матрицы \it A\rm.\\
8. В матрице возвести в квадрат все неотрицательные элементы, отрицательные поделить\linebreak на определитель \it A \rm (реализовать при помощи управляющих конструкций).\\
9. Построить график функции $ \sqrt{\frac{x+3}{x-2}} $ на промежутке [3,5].

\end{document}